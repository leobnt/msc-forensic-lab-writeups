% !TEX TS-program = pdflatex
% !TEX encoding = UTF-8 Unicode

\documentclass[a4paper,
    11pt,
    normalheadings,
%   parskip,
    parindent,
%   draft,
%   final,
%   bibgerm,
%   german,
    UKenglish,
%   twoside,
%   twocolum,
%   openright,  % Kap.beginn immer rechts! (fkt. nur bei report, nicht bei article)
    abstracton,
    ]{scrartcl}
\usepackage[english]{babel}
\usepackage[utf8]{inputenc} % set input encoding (not needed with XeLaTeX)

%%% Examples of Article customizations
% These packages are optional, depending whether you want the features they provide.
% See the LaTeX Companion or other references for full information.

%%% PAGE DIMENSIONS
\usepackage{geometry} % to change the page dimensions
\geometry{a4paper} % or letterpaper (US) or a5paper or....
% \geometry{margins=2in} % for example, change the margins to 2 inches all round
% \geometry{landscape} % set up the page for landscape
%   read geometry.pdf for detailed page layout information

\usepackage{graphicx} % support the \includegraphics command and options

\usepackage[parfill]{parskip} % Activate to begin paragraphs with an empty line rather than an indent

%%% PACKAGES
\usepackage{booktabs} % for much better looking tables
\usepackage{array} % for better arrays (eg matrices) in maths
\usepackage{paralist} % very flexible & customisable lists (eg. enumerate/itemize, etc.)
\usepackage{verbatim} % adds environment for commenting out blocks of text & for better verbatim
\usepackage{subfig} % make it possible to include more than one captioned figure/table in a single float
% These packages are all incorporated in the memoir class to one degree or another...


\usepackage[colorlinks,urlcolor=blue,plainpages=false]{hyperref}
\usepackage{embedfile}        % Provides \embedfile[filename=foo, desc={bar}]{file}
\usepackage{hyperxmp}         % To be have an XMP Data Stream f.e. to include the license

% \usepackage{newcent}  % Different Font, looks bigger

\usepackage{ifthen}

\usepackage[
    T1, % this is good for Umlauts
%   OT1 % This breaks Umlauts
       ]{fontenc}

%%% HEADERS & FOOTERS
\usepackage{fancyhdr} % This should be set AFTER setting up the page geometry
\pagestyle{fancy} % options: empty , plain , fancy
\renewcommand{\headrulewidth}{0pt} % customise the layout...
\lhead{}\chead{}\rhead{}
\lfoot{}\cfoot{\thepage}\rfoot{}

%%% SECTION TITLE APPEARANCE
\usepackage{sectsty}
\allsectionsfont{\sffamily\mdseries\upshape} % (See the fntguide.pdf for font help)
% (This matches ConTeXt defaults)

%%% ToC (table of contents) APPEARANCE
\usepackage[nottoc,notlof,notlot]{tocbibind} % Put the bibliography in the ToC
\usepackage[titles,subfigure]{tocloft} % Alter the style of the Table of Contents
\renewcommand{\cftsecfont}{\rmfamily\mdseries\upshape}
\renewcommand{\cftsecpagefont}{\rmfamily\mdseries\upshape} % No bold!

\usepackage{listings}

%%% END Article customizations

%%% The "real" document content comes below...

% Title Page
\newcommand{\mytitle}{Lab03: Windows Registry}
\subject{Report\\%
    Forensics\\%
    Summersemester 2010\\%
    CA643\\%
    Charlie Daly}
\title{\mytitle{}}
%\subtitle{}
\author{
    cand. Dipl. Inf. Tobias Müller <\href{mailto:muellet2@computing.dcu.ie?subject=ss10-forensic-lab01}{muellet2@}>, 59212333 \and
    BSc. Anthony Walters <\href{mailto:waltera3@computing.dcu.ie?subject=ss10-forensic-lab01}{waltera3@}>, 59213102
    }
% \institute[Mathe - UniHH]{Fachbereich Mathematik\\
%         Universit\"at Hamburg}
% \logo{\includegraphics[height=0.5cm]{cinsects-blue}}
% \titlegraphic{\includegraphics[width=1.5cm]{fsr-logo}}
\date{\today}


\hypersetup{
        pdftitle={\mytitle{}},
        pdfauthor={Tobias Mueller, Anthony Walters},
        pdfsubject={Forensics},
        pdfkeywords={uni, linux, windows, ca643, forensics, registry},
        pdflang=en,
        pdfcopyright={This work is licensed to the public under the Creative Commons Attribution-Non-Commercial-Share Alike 3.0 Germany License.},
        pdflicenseurl={http://creativecommons.org/licenses/by-nc-sa/3.0/de/}
}

\hyphenation{Infor-mations-sys-teme}
\hyphenation{\"{o}ko-logischen}

\newcommand{\note}[1]{#1}
\renewcommand{\comment}[1]{}
\newcommand{\inlinequote}[1]{``\textit{#1}''}
\newcommand{\FIXME}[1]{\mbox{}\marginpar{\footnotesize\raggedright\hspace{0pt}\color{red}\emph{#1}}}

\begin{document}
\selectlanguage{english}
\maketitle


\section{Introduction}
The Windows Registry is a wealthy pool of information.
It is used to store configuration data for system and user applications as well as cached and historical data.
You can thus find paging settings along with, e.g.\, URLs that have been entered in the Internet Explorer recently in the Registry.


In this report, we will examine a given Windows system using the \texttt{reglookup} tool along with GNU tools such as \texttt{grep, find, sed, sort} and \texttt{less} and try to uncover the last actions that were taken on the system.

To begin with, an analysis of Windows registry restore points will be performed and we will try to come to some conclusions on what has been happening  in the system based on the changes we notice in the Restore Points.
We will present a ``registry trail'' which we will follow and comment on the type of information that is stored in the registry for the users convenience.
We will then construct a Registry timeline and merge it with a filesystem timeline, to enable an easier comparison between the two and will then perform some analysis on a portion of the combined timeline.
We also will, where possible, try to use this information in conjunction with information from other relevant Windows artefacts.


\section{Analysis of Windows Registry}
We were given a windows XP dd image which we mounted as follows to preserve its integrity: \verb|sudo mount -o loop,noatime,ro XP-4G.ntfs.dd /mnt/xpdisk/|

Not knowing the Windows platform very well it was interesting to see what concept they introduced to save system and user application data, config and caches.
Fundamentally, the Registry is a big hierarchical database of keys which itself contain multiple key, value pairs.
The correct terms in Registry language are: key, name and data.
A \emph{Key} holds multiple \emph{Names} which in turn hold a strongly typed \emph{Data}.
The types include
    binary,
    string,
    dword
    and
    multi string (very simple list of unique strings)\footnote{A more complete list can be found on Wikipedia: \url{http://en.wikipedia.org/wiki/Windows_Registry}}.
Interestingly, proper list types are missing and it also lacks so called Schemas, which are basically default values for keys that have either not been set by the user or the application or have been deleted.
The database also knows about access control so that a user cannot necessarily modify the system database but rather her personal database only.
As far as we could see,
part of this is implemented using file permissions: The users Registry is per default saved under \texttt{\%USERPROFILE\%\textbackslash{}NTUSER.DAT} and the system level database is saved under \texttt{\%WINDIR\%\textbackslash{}system32\textbackslash{}config\textbackslash{}system} which the user can not write to.
However, we lack a clear understanding of how modifying the Registry actually works on a lower technical level.
Apparently applications are supposed to use an API\footnote{The Windows Advanced API provided by \texttt{advapi32.dll}; documentation, however, is not very verbose.} to modify the Registry but it does not tell how it works internally.
Interesting questions, which may be answered in future experiments, include:
\begin{inparaitem}
    \item If multiple processes write to the same values, how is consistency enforced?
    \item If the Registry is kept per process, what stops us from writing directly to that memory location?
    \item Besides basic write protection described above, Registry keys carry an ACL.
        Hence, some mechanism has to decide whether an user is legitimately trying to write data to a value.
        How is that mechanism implemented?
\end{inparaitem}
Admittedly, these questions are more interesting from an engineering rather than a forensics point of view, but the resulting facts might still be enlightening when examining a Registry.


This Registry approach differs from what, e.g.\, the GNOME desktop does: A central database called GConf\footnote{It will be obsoleted by DConf for GNOME 3.0 (\href{http://live.gnome.org/TwoPointThirtyone/}{to be released on 2010-09-27}) though} exists but in order to modify the key value pairs, the modifying application has to connect to a central server instance (via CORBA or DBus) instead of directly modifying files.
Without providing too much detail (simply because it would be out of scope of this report), this has several advantages such as caching, authentification or preserving integrity.

The Registry can be read using Free Software tools such as \texttt{regviewer}, \texttt{regripper} or \texttt{regloopkup}.
We found \texttt{reglookup} in combination with the powerful GNU userland tools such as \texttt{less}, \texttt{grep} or \texttt{find} to be the best tool to use and an example usage would look like this:
\begin{verbatim}
$ reglookup NTUSER.dat | head
/,KEY,,2009-03-04 21:34:27
/AppEvents,KEY,,2009-03-04 16:43:51
/AppEvents/EventLabels,KEY,,2009-03-04 16:44:09
/AppEvents/EventLabels/.Default,KEY,,2009-03-04 16:43:51
/AppEvents/EventLabels/.Default/,SZ,Default Beep,
/AppEvents/EventLabels/.Default/DispFileName,SZ,@mmsys.cpl\x2C-5824,
/AppEvents/EventLabels/ActivatingDocument,KEY,,2009-03-04 16:44:05
/AppEvents/EventLabels/ActivatingDocument/,SZ,Complete Navigation,
/AppEvents/EventLabels/AppGPFault,KEY,,2009-03-04 16:43:51
\end{verbatim}

The Registry also keeps a timestamp which can be shown, i.e.\, to get an idea of the order of the actions that were taken on the system.
Following command can sort the Registry information using the timestamp as a key:
\begin{verbatim}
$ reglookup -m NTUSER.DAT |  awk -F',' '{ print $4,$1,$2,$3,$5,$6; }' |
                                                              sort | less
\end{verbatim}

Note that the \texttt{-m} switched is not (yet) existing in the currently released (version 0.11) \texttt{reglookup} tool.
We patched \texttt{reglookup} so that it will print the mtime of the parenting key with each value, so that the output is more suitable when sorted by the timestamp.
A patched version can be found and easily installed at the following Ubuntu PPA\footnote{A Personal Package Archive (PPA) is a relatively convenient way of distributing software for Ubuntu users, cmp \href{https://help.launchpad.net/Packaging/PPA}{documentation}.}: \url{https://launchpad.net/~ubuntu-bugs-auftrags-killer/+archive/muelli}.
Executing \texttt{sudo add-apt-repository ppa:ubuntu-bugs-auftrags-killer/muelli} should be sufficient to enable this PPA.
It should then be easy to install the modified \texttt{reglookup} via a \texttt{sudo apt-get install regloopkup}.

As mentionend in our previous report, It would have been nice if there was a standard way of outputting or saving events like file or Registry modification.
With the just mentioned patch we just followed a very naive approach which has problems such as suggesting that the timestamp on a non-key type is actually meaningful.
But since only key types carry a timestamp, it is not.
However, due to timecontraints, the taken approach worked well enough for us.


The semantics of that timestamp on a Registry key, however, are not entirely clear.
As later analysis will show, we can safely assume that, with the exception of a few interesting corner cases, Windows will save the time of the last modification of any values or data under a key.

We can also see the growth patterns of the Registry in question using the standard GNU command line tools on copies of the Registry files kept in the Windows system restore directories.
This information, while strictly not very useful for detailed forensic analysis, can give a general overview of the growth patterns of the Registry in general.
We examined the Registry information which is stored in Windows' restore points but our findings are not too interesting, as the number of keys contantly grew.
In the output below, the first column displays the number of keys found in the Registry file mentioned in the second column.
This usage pattern, shown below, indicates a pretty normal usage of the windows system, with no indications of excessive Registry key deletions (perhaps in order to hide something) over time:
\begin{verbatim}
$ cd System\ Volume\ Information/_restore*/
$ find . -type f -exec  grep -aPobH "nk........\xc9\x01" {} \; |
   grep -aoE "^[^:]+:" | uniq -c | sort -rn | less
  35644 ./RP2/snapshot/_REGISTRY_MACHINE_SOFTWARE:
  35396 ./RP1/snapshot/_REGISTRY_MACHINE_SOFTWARE:
   8720 ./RP3/snapshot/_REGISTRY_MACHINE_SYSTEM:
   8714 ./RP2/snapshot/_REGISTRY_MACHINE_SYSTEM:
   8588 ./RP1/snapshot/_REGISTRY_MACHINE_SYSTEM:
    995 ./RP3/snapshot/_REGISTRY_USER_NTUSER_S...
    980 ./RP2/snapshot/_REGISTRY_USER_NTUSER_S...
    810 ./RP1/snapshot/_REGISTRY_USER_NTUSER_S...
\end{verbatim}














\section{Restore Points Revisited}
Windows Restore Points save the users and the systems Registry files which can of course be examined or compared to earlier or later versions.
To visualise the difference between two Registry files, once could generate two dumps using the above mentioned \texttt{reglookup} tool and then view these dumps in a diff viewer such as \texttt{kdiff3} or simply with \texttt{diff}.

Doing a forensic examination, we are naturally interested in information that the user wanted to get rid of on her active system but which got backed up during creation of a Restore Point.
Hence we were particularly interested in the part of the diff which removed more lines than it added.
To automatically scan for so called hunks in the diff that expose more deleted than added lines, we wrote a small program, shown in \autoref{code:diffalize} that prints the interesting hunks.

\begin{figure}
    \lstinputlisting[language=Python]{../Tools/diffalize/diffalize.py}
    \caption{Code which filters diffs for more removed lines}%
    \embedfile[filespec=diffalize.py, mimetype=text/x-python,desc={diffalize a diff analyzer}]{../Tools/diffalize/diffalize.py}%
    \label{code:diffalize}
\end{figure}

Unfortunately, the hunks of potential interest do not reveal any useful information.
We found values under
\texttt{/Software/Microsoft/Windows/CurrentVersion/Ext/Stats/} being deleted and after a quick look at the data as well as some quick research\footnote{\url{http://technet.microsoft.com/en-us/library/bb490627.aspx}} we simply assumed that Windows did that on its own.
We can also see other other keys where values which were deleted and these might contain interesting data but due to lack of time we followed other traces shown in \autoref{sec:iexplorer},
but for completeness, the found data is shown below.
Note that the entries have to be read being prefixed with \texttt{/Software/Microsoft/Windows/CurrentVersion/Internet Settings/5.0/Cache/}.
\begin{verbatim}
 Cookies/CacheLimit,DWORD,0x00002000,
-Extensible Cache,KEY,,2009-03-04 17:56:34
-Extensible Cache/MSHist012009030420090305,KEY,,2009-03-04 16:44:50
-Extensible Cache/MSHist012009030420090305/CachePath,EXPAND_SZ,%USERPROFILE%\x5CLocal Settings\x5CHistory\x5CHistory.IE5\x5CMSHist012009030420090305\x5C,
-Extensible Cache/MSHist012009030420090305/CachePrefix,SZ,:2009030420090305: ,
-Extensible Cache/MSHist012009030420090305/CacheLimit,DWORD,0x00002000,
-Extensible Cache/MSHist012009030420090305/CacheOptions,DWORD,0x0000000B,
-Extensible Cache/MSHist012009030420090305/CacheRepair,DWORD,0x00000000,
+Extensible Cache,KEY,,2009-03-04 22:27:49
 UserData,KEY,,2009-03-04 17:56:34
\end{verbatim}


Below, we can see data that has been added to the Registry after the last Restore Point was created.
Note that this listing must be read with each Key or Value being prefixed with \texttt{/Software/Microsoft/Windows/CurrentVersion/} which has been omitted for this output to make it fit on the page:
\begin{verbatim}
@@ -4185,7 +4247,12 @@
 Explorer/FileExts/.jpeg/OpenWithProgids/jpegfile,NONE,(null),
-Explorer/FileExts/.jpg,KEY,,2009-03-04 17:59:49
+Explorer/FileExts/.jpg,KEY,,2009-03-04 21:36:16
+Explorer/FileExts/.jpg/OpenWithList,KEY,,2009-03-04 22:26:46
+Explorer/FileExts/.jpg/OpenWithList/a,SZ,iexplore.exe,
+Explorer/FileExts/.jpg/OpenWithList/MRUList,SZ,cba,
+Explorer/FileExts/.jpg/OpenWithList/b,SZ,shimgvw.dll,
+Explorer/FileExts/.jpg/OpenWithList/c,SZ,mspaint.exe,
 Explorer/FileExts/.jpg/OpenWithProgids,KEY,,2009-03-04 17:59:49
\end{verbatim}

We do not know where the \textbf{2009-03-04 17:59:49} exactly comes from (by looking at the MAC times, we do know, however, that this is not the time of installation of the Windows system), but the Key apparently registers file handlers.
We observed that the timestamp of other exotic file extensions carry the very same timestamp.
We thus assume that this is the time the program was run for the first time.
Since there was no handler registered for that filetype, we further conclude that no images of the type \texttt{jpg} have been opened  using the \emph{Donald} account before \textbf{2009-03-04 21:36:16} and that \emph{a} JPG image actually was openend at that time, eventually using Paint.

During the rest of this report, we will investigate on \emph{which} file was actually opened and how it got there.




\section{Following IExplorer Registry Trails} \label{sec:iexplorer}
From looking at the just shown difference between the last backup of the Registry and its currently active version, we know that the user was active at around \textbf{2009-03-04 21:36:16}, we searched for other information on the system that indicated what happened around that time.
As browsing the web is one of the activities that users do regularly, it is reasonable to assume that the user did that as well.
We thus had a closer look at the traces the Internet Explorer leaves in the Registry.

For the users convenience, the Internet Explorer saves (manually) visited URLs as well as filled form entries or other settings such as downloads folder.
This enables the user to quickly find the pages she is interested in or have the browser autocomplete form field that need to be filled regularly (Name, Email Address, \ldots{}).
At the same time, this leaves interesting traces about the pages a user willingly visited or data she entered into webforms.
The Internet Explorer uses the Windows Registry to save and retrieve that data.

The following (shortened) listing shows some information that the Internet Explorer saved.
\begin{verbatim}
$ reglookup -H -p "/Software/Microsoft/Internet Explorer" NTUSER.DAT | head -n4
/,KEY,,2009-03-04 22:28:38
/,SZ,,
/Download Directory,SZ,C:\x5CD&S\x5CDonald\x5CMy Docs\x5CMy Pictures\x5Cduckies,
/Desktop,KEY,,2009-03-04 16:48:52
$
\end{verbatim}

So we see that the download directory has been set to a folder called ``My Pictures/duckies''.
Following that trail, to eventually find out what files were downloaded, we searched for that particular folder in the Registry:

\begin{verbatim}
$ reglookup -H NTUSER.DAT | grep --color=no "My Pictures" |
  sed 's,/Software/Microsoft/Windows/CurrentVersion/Explorer/ComDlg32,SMWCE,g' |
  sed 's,C:\\x5CDocuments and Settings\\x5CDonald\\x5C
                                        My Documents\\x5CMy Pictures\\x5C,DDM,g'
/Software/Microsoft/Internet Explorer/Download Directory,SZ,DDMduckies,
SMWCE/OpenSaveMRU/*/a,SZ,DDMduckies\x5CYellow Ducks_s.jpg,
SMWCE/OpenSaveMRU/*/b,SZ,DDMduckies\x5Crubber-ducks.jpg,
SMWCE/OpenSaveMRU/*/c,SZ,DDMduckies\x5CDucksPage1.jpg,
SMWCE/OpenSaveMRU/*/d,SZ,DDMduckies\x5Crubber-duck-1.jpg,
SMWCE/OpenSaveMRU/*/e,SZ,DDMduckies\x5Crubber-duck-logo.jpg,
SMWCE/OpenSaveMRU/*/f,SZ,DDMduckies\x5Cluvduc_lg.jpg,
SMWCE/OpenSaveMRU/*/g,SZ,DDMduckies\x5Cspace-duck.jpg,
SMWCE/OpenSaveMRU/*/h,SZ,DDMduckies\x5Cducks-787976.jpg,
SMWCE/OpenSaveMRU/*/i,SZ,DDMduckies\x5Csea-of-ducks-300.jpg,
SMWCE/OpenSaveMRU/*/j,SZ,DDMduckies\x5Cducks.jpg,
SMWCE/OpenSaveMRU/jpg/a,SZ,DDMduckies\x5Crubber-ducks.jpg,
SMWCE/OpenSaveMRU/jpg/b,SZ,DDMduckies\x5CDucksPage1.jpg,
SMWCE/OpenSaveMRU/jpg/c,SZ,DDMduckies\x5Crubber-duck-1.jpg,
SMWCE/OpenSaveMRU/jpg/d,SZ,DDMduckies\x5Crubber-duck-logo.jpg,
SMWCE/OpenSaveMRU/jpg/e,SZ,DDMduckies\x5Cluvduc_lg.jpg,
SMWCE/OpenSaveMRU/jpg/f,SZ,DDMduckies\x5Cspace-duck.jpg,
SMWCE/OpenSaveMRU/jpg/g,SZ,DDMduckies\x5Cducks-787976.jpg,
SMWCE/OpenSaveMRU/jpg/h,SZ,DDMduckies\x5Csea-of-ducks-300.jpg,
SMWCE/OpenSaveMRU/jpg/i,SZ,DDMduckies\x5Cducks.jpg,
SMWCE/OpenSaveMRU/jpg/j,SZ,DDMduckies\x5CYellow Ducks_s.jpg,
SMWCE/OpenSaveMRU/png/a,SZ,DDMduckies\x5Cducks.png,
...
$
\end{verbatim}
Hence, a program running as user Donald was made saving files with the above mentioned name.
Out of curiosity, we followed a random file from that list (\texttt{space-ducks.jpg}) and searched what information the Registry saved about it.
We found the Key \linebreak[3]\texttt{/Software/Microsoft/ Windows/CurrentVersion/Explorer/RecentDocs/} holding values that saved the above mentioned filename.
Our hypothesis was that the Windows Explorer saves information about files which are opened with it in that key.
To verify our hypothesis, we ran the Windows Explorer and observed its Registry access via the \emph{Process Monitor} tool.
This experimention lead us to the conclusion that our hypothesis is correct and that files which are opened through the Windows Explorer will indeed be saved in the above mentioned key.

The following list thus contains the files that were most likely openend through the Windows Explorer (note that binary contents was stripped down to ASCII information and that some entries are cut off to make them fit on the page):
\begin{verbatim}
$ reglookup -H \
      -p /Software/Microsoft/Windows/CurrentVersion/Explorer/RecentDocs \
      NTUSER.DAT |
    sed 's,/Software/Microsoft/Windows/CurrentVersion/Explorer,SMWCE,g' |
    sed 's,\\x..,\.,g' | tr -s .
SMWCE/RecentDocs,KEY,,2009-03-04 21:39:25
SMWCE/RecentDocs/0,BINARY,d.u.c.k.s.p.n.g.B.2.ducks.lnk.*.d.u.c.k.s.l.n.k.,
SMWCE/RecentDocs/1,BINARY,d.u.c.k.i.e.s.H.2.duckies.lnk.d.u.c.k.i.e.s.l.n.k.,
SMWCE/RecentDocs/MRUListEx,BINARY,.,
SMWCE/RecentDocs/2,BINARY,r.u.b.b.e.r.-.d.u.c.k.s.j.p.g.X.2.rubber-ducks.lnk.8...
SMWCE/RecentDocs/3,BINARY,D.u.c.k.s.P.a.g.e.1.j.p.g.R.2.DucksPage1.lnk.4...
[...]
SMWCE/RecentDocs/7,BINARY,s.p.a.c.e.-.d.u.c.k.j.p.g.R.2.space-duck.lnk.4...
SMWCE/RecentDocs/8,BINARY,d.u.c.k.s.-.7.8.7.9.7.6.j.p.g.X.2.ducks-787976.lnk...
[...]
SMWCE/RecentDocs/.jpg/4,BINARY,l.u.v.d.u.c._.l.g.j.p.g.N.2.luvduc_lg.lnk....
SMWCE/RecentDocs/.jpg/5,BINARY,s.p.a.c.e.-.d.u.c.k.j.p.g.R.2.space-duck.lnk....
SMWCE/RecentDocs/.jpg/6,BINARY,d.u.c.k.s.-.7.8.7.9.7.6.j.p.g.X.2.ducks-787976.lnk...
$
\end{verbatim}

So we can see that a file named e.g.\, \textbf{space-ducks.jpg} was recently openend via the Windows Explorer.

To make more sense out of this data, we will combine our (potentially not meaningful) timestamps from the registry with MAC times from the filesystem.





\section{Combining Registry with Filesystem MAC times}

In this case, a timeline generated from the filesystem and a timeline generated from the users Registry file \texttt{NTUSER.DAT} was combined to produce a single unified timeline for analysis.
It is easier to see the last modified times of Registry keys and how they relate to filesystem creation, modification and access times for files and directories.
What we will be looking for here is some sort of correlation to modified Registry entries and the access times of applications themselves, and will then try to deduce a sequence of events based on this combined timeline.


To make the filesystem timeline, we needed to  slightly modify the command, that we used in previous reports, in a way that the format of the timestamp matches the Registry timeline date format, so that sorting became possible.
A timeline of the filesystem was made as follows from our mounted Windows XP image. Note the command was split onto multiple lines for inclusion in this report.
\begin{verbatim}
find . -printf "%AY-%Am-%Ad %AT (a) %p\n\
                %TY-%Tm-%Td %TT (m) %p\n\
                %CY-%Cm-%Cd %CT (c) %p\n" |
                                            sort > ~/lab3/fstimeline.txt
\end{verbatim}

A Registry timeline was created from the \texttt{NTUSER.DAT} file for the user Donald from the ``./Documents and Settings/Donald'' directory.
A longer version of the Registry timeline was produced by using the previously described \texttt{-m} switch with the reglookup utility.
This added extra output to the Registry timeline to include the values of the Registry keys as well as the timestamped keys themselves.
\begin{verbatim}
reglookup -m NTUSER.DAT | gawk -F , '{print $4, $0}' |
      sort > ~/lab3/regtimeline.txt
\end{verbatim}

Now that we have both the filesystem and user Registry timelines saved and both using the same time format, they can be combined and sorted in the \texttt{combinedtimeline.txt} file for viewing as follows.
\begin{verbatim}
$ cat regtimeline.txt fstimeline.txt | sort > combinedtimeline.txt
\end{verbatim}


In the combined timeline below, the two types of entries can be distinguished as follows.
Filesystem lines will have a date and time followed by a (c) (m) or (a), and Registry entries will not have these timestamp labels.
The times associated with the Registry entries are the last modified timestamp for the \emph{Registry key} and as such, the timestamp values do not hold much meaning for the Registry entries, but are included here for examination.

Below you can see an example of the combined timeline for any lines relating to the file \texttt{space-duck}.
\begin{verbatim}
$ cat combinedtimeline.txt | grep space-duck |
     sed 's,/Software/Microsoft/Windows/CurrentVersion/Explorer,SMWCE,g' |
          sed 's,\\x..,\.,g' | tr -s .
2009-03-04 21:37:05.0000000000 (c)
     ./Documents and Settings/Donald/My Documents/My Pictures/duckies/space-duck.jpg
2009-03-04 21:37:05.0000000000 (m)
     ./Documents and Settings/Donald/My Documents/My Pictures/duckies/space-duck.jpg
2009-03-04 21:37:45.0000000000 (a) ./Documents and Settings/Donald/Recent/space-duck.lnk
2009-03-04 21:37:45.0000000000 (c) ./Documents and Settings/Donald/Recent/space-duck.lnk
2009-03-04 21:37:45.0000000000 (m) ./Documents and Settings/Donald/Recent/space-duck.lnk
2009-03-04 21:38:11.0000000000 (a)
     ./Documents and Settings/Donald/My Documents/My Pictures/duckies/space-duck.jpg
2009-03-04 21:39:25 SMWCE/ComDlg32/OpenSaveMRU/*/g,SZ,
     C:.Documents and Settings.Donald.
          My Documents.My Pictures.duckies.space-duck.jpg,2009-03-04 21:39:25
2009-03-04 21:39:25 SMWCE/ComDlg32/OpenSaveMRU/jpg/f,SZ,
     C:.Documents and Settings.Donald.My Documents.
          My Pictures.duckies.space-duck.jpg,2009-03-04 21:39:25
2009-03-04 21:39:25 SMWCE/RecentDocs/7,BINARY,
     s.p.a.c.e.-.d.u.c.k.j.p.g.R.2.space-duck.lnk.4.s.p.a.c.e.-.d.u.c.k.l.n.k.,2009-03-04 21:39:25
2009-03-04 21:39:25 SMWCE/RecentDocs/.jpg/5,BINARY,
     s.p.a.c.e.-.d.u.c.k.j.p.g.R.2.space-duck.lnk.4.s.p.a.c.e.-.d.u.c.k.l.n.k.,2009-03-04 21:39:25
\end{verbatim}

From the above timeline, we can compare the relative times for each entry to deduce a probable sequence of events for the \texttt{space-duck} image file.

First of all, you can see that the space-duck.jpg file appeared in the filesystem at \textbf{21:37:05}. So some action created the image file in the filesystem at this time.
You can see that the next reference to the space-duck file is fourty seconds later in the form of a windows link file being created.
You might conclude from this that the \texttt{space-duck.jpg} file was browsed to and doubleclicked at \textbf{21:37:45}, thus creating the windows \texttt{.lnk} file.

The next thing to notice in this timeline is that at \textbf{21:38:11}, 26 seconds after the link files were created, the access time access time of the \texttt{space-duck.jpg} file was updated.
So taking this difference in timestamps, this may indicate that the \texttt{space-duck.jpg} file was double-clicked and an application which took 26 seconds to start up then eventually opened the \texttt{space-duck.jpg} file and so updating it's access time.
Note that we had a look at the context that surrounds the above mentioned timestamps but we did not find anything interesting so we did not consider including it.

The last thing to note in this timeline is the \texttt{OpenSaveMRU/jpg} and \texttt{RecentDocs.jpg} registry key timestamps (\textbf{21:39:25}) which reference the \texttt{space-duck} image file.
As explained earlier, these timestamps are not particularly useful in this instance as they indicate the last time \emph{any file} with the extention \texttt{jpg} was opened and so cannot be used to make any conclusion here about the \texttt{space-duck} image file.

So in summary the timestamps in this timeline indicate a reasonable sequence of events which may have happened during the last windows session; that the image was first created, then browsed to through windows explorer, double-clicked and after a short time, an application starts and opens the image file itself.


\section{Conclusion}

We have shown how to use the \texttt{reglookup} tool, which we modified to meet our needs, to generate a timeline and unify that with an already existing event information, i.e.\, MAC times from a filesystem.
%
We have also discussed the ambigious semantic of the timestamp saved in the Registry as well as stated our problems understanding how exactly all the interesting parts of the Registry work, i.e.\, Integrity or Authorisation.
%
By analysing the Restore Points we identified a regular usage pattern of the Registry.
%
During this analysis we wrote a simple tool called \texttt{diffalize} that extracts potentially interesting changes in the Registry.
%
By identifying and following various Registry Keys which are related to the Windows Explorer and the Internet Explorer we could show that an image was most likely downloaded, actively saved and viewed.
%
After incorporating other time based event information (filesystems MAC times) into the Registry timeline, we finally showed which changes in the Registry are potentially related to changes or access on the disk.



\section*{License}
This work is licensed to the public under the Creative Commons Attribution-Non-Commercial-Share Alike 3.0 Germany License.
\begin{center}\includegraphics{bin/by-nc-sa-eu.png}\end{center}


\end{document}